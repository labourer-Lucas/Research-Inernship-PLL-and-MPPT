\setcounter{equation}{0}

\clearpage

\chapter{Algorithm Development for MPPT} \label{chapter:Algorithm Development for MPPT}
\section{Overview of MPPT and its Applications}
Maximum Power Point Tracking (MPPT) is a family of control algorithms that aims at optimizing the use of a power source that possesses a fluctuating power profile.

Indeed, some power sources, like solar panels, present power characteristics that strongly depend on the operating conditions. For instance, the cloud coverage significantly impacts the capability of a panel to deliver electricity. As such, maximizing the extracted power requires identifying and tracking the operating point that provides the highest power level as a function of the operating conditions.

Therefore, MPPT is often applied in renewable energy systems, e.g. photovoltaic plants or wind turbines, as their power delivery capability varies significantly and in an unpredictable manner. Other special operating points may be interesting to track, such as the maximum efficiency point tracking (MEPT), or other optimum, e.g. related to operating costs.

\section{Mathematical Model of PV Array}
The PV array is the largest building block of the PV system which consists of
 PV panels, then PV modules. The PV modules are consisting of several PV cells
 connected in series and parallel to produce the required voltage and current from the
 module. So, the PV cell is the basic unit of the PV systems.  The PV cell is consisting
 of two semiconductor materials from types P and N. The PN junction absorbs the
 light from the Sun which adds energy to the electrons in this junction enabling it to
 have enough energy to cross the junction and produce voltage difference between
 their terminals.  

 There are several research works that have been done to model the PV cell\cite{6365813}. The one-diode model is the most common model used to represent the electrical characteristics of the PV cell. The equivalent circuit of the one-diode model is shown in Fig.\ref{fig:one-diode-model}.
 \begin{figure}[H]
	\centering
	\includegraphics[width=0.6\textwidth]{Chapter_3/Graphics/one-diode-model.png}
	\caption{The equivalent circuit of the one-diode model of a PV cell.}
	\label{fig:one-diode-model}
\end{figure}

Moreover, the one-diode model parameters can be easily determined experimentally\cite{MA201431}. Some other models like the two-diode model\cite{7324622} and the three-diode model\cite{KHANNA2015105} have been proposed to represent the PV cell more accurately, but they are more complex. Therefore, we just discuss the one-diode model in this report.

 The output current generated from the PV cell is given by the following equation:
 \begin{equation}
 	I_{\mathrm{PVC}}=I_{\mathrm{LG}}-I_{\mathrm{sat}}*\left[e^{\left(\frac{q}{KT}(V_{\mathrm{PVC}}+R_{s}I_{\mathrm{PVC}})\right)}-1\right]-\frac{V_{\mathrm{PVC}}+R_{s}I_{\mathrm{PVC}}}{R_{\mathrm{sh}}}
	 \end{equation}
	  where:
	   \begin{itemize}
		\item $I_{\mathrm{LG}}$: Light-generated current (A)
		\item $I_{\mathrm{sat}}$: Reverse saturation current (A)
		\item $q$: Charge of an electron (C)
		\item $K$: Boltzmann constant (J/K)
		\item $T$: Absolute temperature (K)
		\item $V_{\mathrm{PVC}}$: Output voltage of the PV cell (V)
		\item $R_{s}$: Series resistance ($\Omega$)
		\item $R_{\mathrm{sh}}$: Shunt resistance ($\Omega$)
	\end{itemize}

	 The light-generated current for given radiation and temperature can be calculated as follows:
	 \begin{equation}
		I_{LG}=(I_{\mathrm{STC}}+K_I(T_c-T_r))\frac{G}{G_o}
	\end{equation}
	  where:
	   \begin{itemize}	
		\item $I_{\mathrm{STC}}$: PV cell current at standard test conditions (A)
		\item $K_I$: Temperature coefficient of current (A/K)
		\item $T_c$: PV cell temperature (K)			
		\item $T_r$: Rated temperature (K)
		\item $G$: The current solar irradiance (W/m²)
		\item $G_o$: Reference solar irradiance (kW/m²)
	\end{itemize}

	The module voltage can be obtained by multiplying the cell voltage by the number of cells connected in series $N_{SC}$ in the module.

	\begin{equation}
		V_{\mathrm{M}}=N_{SC}*V_{\mathrm{PVC}}
	\end{equation}

	The module current can be obtained by multiplying the cell current by the number of cells connected in parallel $N_{PC}$ in the module.

	\begin{equation}
		I_{\mathrm{M}}=N_{PC}*I_{\mathrm{PVC}}
	\end{equation}

	Multiplying the terminal voltage by the output current determines the generated
 power from the PV array. The relation between the terminal voltage and current in
 uniform condition for different irradiances, and the relation between the terminal
 voltage and output power are shown in Fig.\ref{fig:PV-characteristics}.
 \begin{figure}[H]
	\centering
	\includegraphics[width=0.6\textwidth]{Chapter_3/Graphics/PV-characteristics.png}
	\caption{The characteristics of a PV module under different irradiance levels.}
	\label{fig:PV-characteristics}
\end{figure}

It's clear that the PV module has a nonlinear characteristic, and there is a unique point on each curve where the power is maximum. This point is called the Maximum Power Point (MPP), and the goal of the MPPT algorithms is to track this point under different operating conditions.



\section{Perturb and Observe Algorithm}
The Perturb and Observe (P\&O) algorithm is one of the most widely used MPPT techniques due to its simplicity and ease of implementation. The basic idea behind the P\&O algorithm is to perturb the operating voltage of the PV module and observe the resulting change in power output. If the power increases, the perturbation is continued in the same direction; if the power decreases, the perturbation direction is reversed. This process is repeated continuously to track the MPP\cite{TAFTICHT20081508}.

The flowchart of the P\&O algorithm is shown in Fig.\ref{fig:PO-flowchart}.
 \begin{figure}[H]
	\centering
	\includegraphics[width=0.6\textwidth]{Chapter_3/Graphics/PO-flowchart.png}
	\caption{The flowchart of the Perturb and Observe algorithm.}
	\label{fig:PO-flowchart}
\end{figure}

There are many P\&O techniques that have been proposed to improve the performance of the basic P\&O algorithm. Some of these techniques include adaptive step size and variable perturbation frequency\cite{6398612}. 
\subsection{Response of PV Module to Irradiance Variation}
When the irradiance level changes suddenly, the output power of the PV module also changes. The response of the PV module to a sudden change in irradiance from 1000 W/m² to 500 W/m² is shown in Fig.\ref{fig:PV-irradiance-response}.
 \begin{figure}[H]
	\centering
	\includegraphics[width=0.6\textwidth]{Chapter_3/Graphics/PV-irradiance-response.png}
	\caption{The response of a PV module to a sudden change in irradiance with MPPT}
	\label{fig:PV-irradiance-response}
\end{figure}

\section{Incremental Conductance Algorithm}
The Incremental Conductance (IncCond) algorithm is another popular MPPT technique that is based on the principle that the derivative of power with respect to voltage is zero at the MPP. The IncCond algorithm calculates the incremental conductance ($dI/dV$) and compares it to the instantaneous conductance ($I/V$) to determine the direction of perturbation. If $dI/dV > -I/V$, the operating voltage is increased; if $dI/dV < -I/V$, the operating voltage is decreased. This process is repeated continuously to track the MPP\cite{6733340}.

The flowchart of the IncCond algorithm is shown in Fig.\ref{fig:IncCond-flowchart}.
 \begin{figure}[H]
	\centering
	\includegraphics[width=0.6\textwidth]{Chapter_3/Graphics/IncCond-flowchart.png}
	\caption{The flowchart of the Incremental Conductance algorithm.}
	\label{fig:IncCond-flowchart}
\end{figure}

\subsection{Response of PV Module to Irradiance Variation}
When the irradiance level changes suddenly, the output power of the PV module also changes. The response of the PV module to a sudden change in irradiance from 1000 W/m² to 500 W/m² is shown in Fig.\ref{fig:PV-irradiance-response-IncCond}.
 \begin{figure}[H]
	\centering
	\includegraphics[width=0.6\textwidth]{Chapter_3/Graphics/PV-irradiance-response-IncCond.png}
	\caption{The response of a PV module to a sudden change in irradiance with MPPT using Incremental Conductance algorithm.}
	\label{fig:PV-irradiance-response-IncCond}
\end{figure}

\section{Comparison between P\&O and IncCond Algorithms}
\begin{itemize}
	\item P\&O: A simple MPPT algorithm that perturbs the operating voltage slightly and observes the effect on output power. If power increases, it continues in that direction; if it decreases, it reverses.
	\item IncCond: A more advanced method that calculates the derivative of power with respect to voltage (dP/dV) to predict whether the system is approaching or moving away from the MPP.
\end{itemize}

The main differences between the two algorithms are summarized in Table \ref{tab:PO-vs-IncCond}.
\begin{table}[H]
\centering
\caption{Comparison between P\&O and IncCond Algorithms}
\label{tab:PO-vs-IncCond}
\begin{tabular}{|p{3cm}|p{4cm}|p{4cm}|}
\hline
\multicolumn{1}{|c|}{\textbf{Feature}} & \multicolumn{1}{c|}{\textbf{P\&O}}                                              & \multicolumn{1}{c|}{\textbf{IncCond}}      \\ \hline
\textbf{Complexity}                    & Simple and easy to implement                                                    & More complex (needs derivative calculations)                         \\ \hline
\textbf{Cost}                          & Low computational cost                                                          & Higher computational cost                                            \\ \hline
\textbf{Tracking Accuracy}             & Less accurate, especially under fast-changing irradiance                        & More accurate, especially in dynamic conditions                      \\ \hline
\textbf{Oscillations at MPP}           & Yes – tends to oscillate around MPP                                             & Minimal – can settle exactly at MPP                                  \\ \hline
\textbf{Speed of Convergence}          & Moderate                                                                        & Faster in dynamic conditions                                         \\ \hline
\textbf{Response to Irradiance Change} & Can be confused by rapid changes – may mistake irradiance shifts for MPP shifts & Handles irradiance changes more accurately                           \\ \hline
\textbf{Implementation}                & Widely used in small-scale systems due to simplicity                            & Preferred in systems where accuracy and efficiency are more critical \\ \hline
\textbf{Sensitivity to Noise}          & Low                                                                             & Can be more sensitive due to derivative calculations                 \\ \hline
\end{tabular}
\end{table}