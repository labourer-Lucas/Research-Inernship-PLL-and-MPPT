\setcounter{equation}{0}

\clearpage
\chapter{Design and Implementation} \label{chapter:Design and Implementation}
To implement the Model Based Design (MBD) for the PLL algorithms, MATLAB/Simulink is used as the simulation platform.
\section{Modeling in MATLAB/Simulink for PLL}
 The following subsections describe the different PLL models developed in MATLAB/Simulink.
\subsection{Test Signal Generation Block}
The test signal generation block is used to generate the three-phase voltage signals with adjustable frequency, amplitude, phase angle, negative sequence, and harmonic distortion. The fundamental waveform generation block is shown in Fig.~\ref{fig:Fundamental_Waveform_Generation_Block}. 
\begin{figure}[H]
    \centering
    \includegraphics[width=0.6\textwidth]{Chapter_4/Graphics/Fundamental_Waveform_Generation_Block.png}
    \caption{Fundamental Waveform Generation Block}
    \label{fig:Fundamental_Waveform_Generation_Block}
\end{figure}

The harmonic waveform generation block is shown in Fig.~\ref{fig:Harmonic_Waveform_Generation_Block}.
\begin{figure}[H]
    \centering
    \includegraphics[width=0.6\textwidth]{Chapter_4/Graphics/Harmonic_Waveform_Generation_Block.png}
    \caption{Harmonic Waveform Generation Block}
    \label{fig:Harmonic_Waveform_Generation_Block}
\end{figure}

The negative sequence waveform generation block is shown in Fig.~\ref{fig:Negative_Sequence_Waveform_Generation_Block}.
\begin{figure}[H]
    \centering
    \includegraphics[width=0.6\textwidth]{Chapter_4/Graphics/Negative_Sequence_Waveform_Generation_Block.png}
    \caption{Negative Sequence Waveform Generation Block}
    \label{fig:Negative_Sequence_Waveform_Generation_Block}
\end{figure}
\subsection{SRF-PLL Model}
The SRF-PLL model developed in MATLAB/Simulink is shown in Fig.~\ref{fig:SRF_PLL_Model}.
\begin{figure}[H]
    \centering
    \includegraphics[width=1\textwidth]{Chapter_4/Graphics/SRF_PLL_Model.png}
    \caption{SRF-PLL Model}
    \label{fig:SRF_PLL_Model}
\end{figure}

The SRF-PLL model consists of the blocks such as Clarke Transformation, Park Transformation, PI Controller with saturation limits and a foward Euler integrator to obtain the phase angle. The model is then used to implement MBD and generate C code for real-time implementation. 
\subsection{DDSRF-PLL Model}
The double decoupled block is shown in Fig.~\ref{fig:Double_Decoupled_Block}. It extracts the positive and negative sequence components and subtracts them from the original signal to provide a decoupled signal for the SRF-PLL. The block implements the Equantion \ref{eq:decoupled_positive_sequence} mentioned in Chapter~\ref{chapter:Algorithm Development of Phase-Locked Loop}.
\begin{figure}[H]
    \centering
    \includegraphics[width=0.5\textwidth]{Chapter_4/Graphics/Double_Decoupled_Block.png}
    \caption{Double Decoupled Block}
    \label{fig:Double_Decoupled_Block}
\end{figure}

The complete DDSRF-PLL model developed is shown in Fig.~\ref{fig:DDSRF_PLL_Model}.
\begin{figure}[H]
    \centering
    \includegraphics[width=1\textwidth]{Chapter_4/Graphics/DDSRF_PLL_Model.png}
    \caption{DDSRF-PLL Model}
    \label{fig:DDSRF_PLL_Model}
\end{figure}
After the block is a SRF-PLL block similar to the one shown in Fig.~\ref{fig:SRF_PLL_Model}.
\subsection{SOGI-PLL Model}
We use Tustin discretization method for the SOGI integrator implementation in the SOGI-PLL model. The Tustin method provides a good balance between accuracy and stability for the integrator in this application, and its block diagram is shown in Fig.~\ref{fig:Tustin_Integrator_Block}.
\begin{figure}[H]
    \centering
    \includegraphics[width=0.9\textwidth]{Chapter_4/Graphics/Tustin_Integrator_Block.png}
    \caption{Tustin Integrator Block}
    \label{fig:Tustin_Integrator_Block}
\end{figure}

The SOGI block used in the SOGI-PLL model is shown in Fig.~\ref{fig:SOGI_Block}.    
\begin{figure}[H]
    \centering
    \includegraphics[width=0.8\textwidth]{Chapter_4/Graphics/SOGI_Block.png}
    \caption{SOGI Block}
    \label{fig:SOGI_Block}
\end{figure}

If the DC offset rejection is required, an additional integrator is needed in the SOGI block as shown in Fig.~\ref{fig:SOGI_Block_with_DC_Offset_Rejection}.
\begin{figure}[H]
    \centering
    \includegraphics[width=0.8\textwidth]{Chapter_4/Graphics/SOGI_Block_with_DC_Offset_Rejection.png}
    \caption{SOGI Block with DC Offset Rejection}
    \label{fig:SOGI_Block_with_DC_Offset_Rejection}
\end{figure}
% The complete SOGI-PLL model developed is shown in Fig.~\ref{fig:SOGI_PLL_Model}.
% \begin{figure}[H]
%     \centering
%     \includegraphics[width=1\textwidth]{Chapter_4/Graphics/SOGI_PLL_Model.png}
%     \caption{SOGI-PLL Model}
%     \label{fig:SOGI_PLL_Model}
% \end{figure}
\subsection{DSOGI-PLL Model}
DSOGI uses two SOGI blocks to extract the positive and negative sequence components separately. The block diagram of the DSOGI is shown in Fig.~\ref{fig:DSOGI_Block}.
\begin{figure}[H]
    \centering
    \includegraphics[width=0.6\textwidth]{Chapter_4/Graphics/DSOGI_Block.png}
    \caption{DSOGI Block}
    \label{fig:DSOGI_Block}
\end{figure}

The rest of the DSOGI-PLL model is similar to the SRF-PLL model.
\subsection{MSOGI-PLL Model}
MSOGI uses multiple SOGI blocks to extract the harmonic components separately. The block diagram of the MSOGI is shown in Fig.~\ref{fig:MSOGI_Block}.
\begin{figure}[H]
    \centering
    \includegraphics[width=0.8\textwidth]{Chapter_4/Graphics/MSOGI_Block.png}
    \caption{MSOGI Block}
    \label{fig:MSOGI_Block}
\end{figure}
The rest of the MSOGI-PLL model is similar to the SRF-PLL model.
\section{Modeling in MATLAB/Simulink for MPPT}
    The following subsections describe the different MPPT models developed in MATLAB/Simulink.
\subsection{MPPT P\&O Model}
The MPPT P\&O model developed in MATLAB/Simulink is shown in Fig.~\ref{fig:MPPT_PO_Model}.
\begin{figure}[H]
    \centering
    \includegraphics[width=1\textwidth]{Chapter_4/Graphics/MPPT_P&O_Model.png}
    \caption{MPPT P\&O Model}
    \label{fig:MPPT_PO_Model}
\end{figure}

\subsection{MPPT IncCond Model}
The MPPT IncCond model developed in MATLAB/Simulink is shown in Fig.~\ref{fig:MPPT_IncCond_Model}.
\begin{figure}[H]
    \centering
    \includegraphics[width=1\textwidth]{Chapter_4/Graphics/MPPT_IncCond_Model.png}
    \caption{MPPT IncCond Model}
    \label{fig:MPPT_IncCond_Model}
\end{figure}

\cleardoublepage
